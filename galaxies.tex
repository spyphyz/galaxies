\documentclass{article}
%sets paragraph indent for whole document
\setlength{\parindent}{0pt}
%gap between paragraphs
\setlength{\parskip}{1em}
%line spacing (1.3 is equivalent to one and a half spacing)
\renewcommand{\baselinestretch}{1.3}
\usepackage[english]{babel}

% Set page size and margins
% Replace `letterpaper' with`a4paper' for UK/EU standard size
\usepackage[a4paper,top=2cm,bottom=2cm,left=3cm,right=3cm,marginparwidth=1.75cm]{geometry}

% Useful packages
\usepackage{amsmath}
\usepackage{graphicx}
\usepackage[colorlinks=true, allcolors=blue]{hyperref}


\author{Dr Simon P.~Young \thanks{MSc Student, LJMU, UK}}
\title {Galaxies}


\begin{document}
% generate the title
\maketitle


\begin{abstract}
The author takes no responsibilities for mistakes! Where possible, equations are checked for accuracy using more than one source.
\end{abstract}

\tableofcontents

\newpage

\section {Topic Overview}

\begin{itemize}
  \item Distance Measures
  \item Colour-Magnitude Diagram
  \item Initial Mass Function (IMF), including Salpeter
  \item Virial Theorem, and the log-sigma - log-Re plane
  \item Galaxy Luminosity Function, and the Schechter Function
  \item Morphology of Galaxies
  \item Sersec and De Vaucolouers Profiles
  \item Feedback
  \item Star formation and the Kennicut-Schmidt Law
  \item Stability - Jeans' and (Safronov-)Toomre Stability Criterion
  \item Density Waves, incl Lin-Shu
  \item Magnitudes
  \item AGNs
  \item Black holes
  \item Dark Matter and Rotation Curves

\end{itemize}

\newpage
\textbf{Colour-Magnitude Diagram of Galaxies}
\begin{itemize}
  \item Blue Sequence
  \item Red Cloud
  \item Green Valley
\end{itemize}


\section {Distances}

\text{Parallax Space Missions:}
\begin{itemize}
    \item Tully Fisher
\end{itemize}


\newpage

\section {Sources}
Gaia - \url{https://sci.esa.int/web/gaia/home}
\end{document}





  